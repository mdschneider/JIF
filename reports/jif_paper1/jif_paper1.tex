%
%  jif_paper1.tex
%
%  Created by Michael D. Schneider on 2015-04-27
%    <schneider@ucdavis.edu>
%
\documentclass[11pt, letterpaper]{article}

\usepackage[OT1]{fontenc}
\usepackage[in]{fullpage}

\usepackage[pdftex,pdfpagemode={UseOutlines},bookmarks,bookmarksopen,colorlinks,linkcolor={black},citecolor={black},urlcolor={red}]{hyperref}
\usepackage{graphicx}
\usepackage{bm}
\usepackage{amsmath,amssymb}
\usepackage{aas_macros}

\usepackage{mathptmx}  % times roman, including math (where possible)

% Commands for colorful collaborative markup
% (copied from http://www.astrobetter.com/whats-the-best-tool-to-annotate-pdfs/#comment-10303)
% \usepackage[usenames]{color}
% \newcommand{\mike}[1]{\textcolor{Red}{\bf #1}}
% \newcommand{\collaborator}[1]{\textcolor{ForestGreen}{\bf #1}}

% \setlength{\parskip}{6pt}

% Macros
\newcommand{\half}{\frac{1}{2}}
\newcommand{\rhocrit}{\rho_{\rm crit}}
\newcommand{\rvir}{r_{\rm vir}}
\newcommand{\mvir}{m_{\rm vir}}
\newcommand{\om}{\Omega_{m}} 
\newcommand{\kv}{\mathbf{k}}
\newcommand{\xv}{\mathbf{x}}
\newcommand{\hmsun}{h^{-1}M_{\odot}}
\newcommand{\hmpc}{h^{-1}Mpc}
\newcommand{\hgpc}{h^{-1}{\rm Gpc}}


\begin{document}
  
\title{Cosmic shear measurements from space and ground: \\Systematics from non-overlapping wavelengths}

\author{JIF collaborators}
%$^{1} Livermore National Laboratory, P.O. Box 808 L-210, Livermore, CA 94551-0808, USA.

\date{\today}

\maketitle
% \tableofcontents

\begin{abstract}
	Galaxy morphologies can appear different at different wavelengths because distinct regions of 
	a galaxy can have different spectral energy distributions (SEDs). In this paper we demonstrate an 
	optimal algorithm to combine galaxy imaging data in different wavelength passbands to measure 
	the gravitational lensing distortions of galaxy shapes, or `cosmic shear.'
	We focus on the combination of space-based and ground-based imaging, which is
	further complicated by differing point-spread functions, pixel sizes, and detector characteristics.
	Assuming perfect knowledge of the SEDs of all components of a galaxy image, we show that a 
	full forward model of all joint imaging data improves shear inferences by a factor of XXX. 
	Assuming instead only weak prior information on galaxy SED components, the shear inference from 
	combined imaging from the Large Synoptic Survey Telescope (LSST) and WFIRST-AFTA would be improved 
	by a factor of YYY if WFIRST-AFTA contained an optical passband overlapping those of LSST.
\end{abstract}

% -----------------------------------------------------------------------------
\section{Introduction} % (fold)
\label{sec:introduction}

% section introduction (end)


% -----------------------------------------------------------------------------
\section{Conclusions} % (fold)
\label{sec:conclusions}

% section conclusions (end)

\end{document}

%
%  jif_paper1.tex
%
%  Created by Michael D. Schneider on 2015-04-27
%    <schneider@ucdavis.edu>
%
\documentclass[11pt, letterpaper]{article}

\usepackage[OT1]{fontenc}
\usepackage[in]{fullpage}

\usepackage[pdftex,pdfpagemode={UseOutlines},bookmarks,bookmarksopen,colorlinks,linkcolor={black},citecolor={black},urlcolor={red}]{hyperref}
\usepackage{graphicx}
\usepackage{bm}
\usepackage{amsmath,amssymb}
\usepackage{aas_macros}

\usepackage{mathptmx}  % times roman, including math (where possible)

% Commands for colorful collaborative markup
% (copied from http://www.astrobetter.com/whats-the-best-tool-to-annotate-pdfs/#comment-10303)
% \usepackage[usenames]{color}
% \newcommand{\mike}[1]{\textcolor{Red}{\bf #1}}
% \newcommand{\collaborator}[1]{\textcolor{ForestGreen}{\bf #1}}

% \setlength{\parskip}{6pt}

\renewcommand*\contentsname{Agenda}

% Macros
\newcommand{\half}{\frac{1}{2}}
\newcommand{\rhocrit}{\rho_{\rm crit}}
\newcommand{\rvir}{r_{\rm vir}}
\newcommand{\mvir}{m_{\rm vir}}
\newcommand{\om}{\Omega_{m}} 
\newcommand{\kv}{\mathbf{k}}
\newcommand{\xv}{\mathbf{x}}
\newcommand{\hmsun}{h^{-1}M_{\odot}}
\newcommand{\hmpc}{h^{-1}Mpc}
\newcommand{\hgpc}{h^{-1}{\rm Gpc}}


\begin{document}
  
\title{Schneider update\\
Meeting notes for 2015-04-27 }

\author{Michael D. Schneider}
%$^{1} Livermore National Laboratory, P.O. Box 808 L-210, Livermore, CA 94551-0808, USA.

\date{\today}

\maketitle
\tableofcontents

\end{document}

